\documentclass[12pt,a4paper]{article}

% Pacotes úteis
\usepackage[utf8]{inputenc}   % Acentos
\usepackage[T1]{fontenc}
\usepackage{graphicx}         % Figuras
\usepackage{array}            % Tabelas mais bonitas
\usepackage{longtable}        % Tabelas grandes
\usepackage{geometry}         % Margens
\usepackage{rotating}

\geometry{a4paper, margin=2cm}

% Informações de capa
\title{Relatório de Verificação Funcional \\ 
Decoder Scancode para ASCII}
\author{Bernardo Farinon}
\date{\today}

\begin{document}

\maketitle

\section{Introdução}
Este relatório apresenta os resultados da verificação funcional do circuito lógico 
CUV, fornecido pelo professor, 
comparado a um modelo de referência funcional (Golden Model)
desenvolvido de acordo com a especificação do projeto. 

O objetivo foi validar se o CUV implementa corretamente a conversão de códigos 
\textit{scancode} para \textit{ASCII}, considerando apenas os caracteres alfanuméricos 
(A--Z e 0--9). Entradas fora desse conjunto devem resultar em \texttt{0xFF}. 

\section{Metodologia}
\begin{itemize}
    \item Foi desenvolvido um Golden Model em VHDL, implementando a conversão direta 
    de scancodes para seus equivalentes ASCII.
    \item Foi criado um testbench que aplica todos os valores possíveis 
    (\texttt{0x00} a \texttt{0xFF}) simultaneamente no CUV e no Golden.
    \item As saídas de ambos foram comparadas automaticamente durante a simulação no QuestaSim.
    \item Divergências foram registradas em uma tabela de verificação.
\end{itemize}

\section{Resultados da Verificação}
A Tabela~\ref{tab:verificacao} mostra as divergências encontradas entre o CUV 
e o Golden Model. Para cada caso, são apresentados: a entrada em hexadecimal e decimal, 
o resultado esperado (Golden), e o resultado obtido (CUV).

{\scriptsize
\begin{longtable}{|c|c|c|c|c|c|}
\hline
\textbf{Entrada (HEX)} & \textbf{Entrada (DEC)} & 
\textbf{Esperado (HEX)} & \textbf{Esperado (DEC)} & 
\textbf{Obtido (HEX)} & \textbf{Obtido (DEC)} \\
\hline
\endfirsthead
\hline
\textbf{Entrada (HEX)} & \textbf{Entrada (DEC)} & 
\textbf{Esperado (HEX)} & \textbf{Esperado (DEC)} & 
\textbf{Obtido (HEX)} & \textbf{Obtido (DEC)} \\
\hline
\endhead
0x14 & 20  & 0xFF & 255 & 0x09 & 9   \\ \hline
0x20 & 32  & 0xFF & 255 & 0x03 & 3   \\ \hline
0x21 & 33  & 0x43 & 67  & 0x03 & 3   \\ \hline
0x2B & 43  & 0x46 & 70  & 0x66 & 102 \\ \hline
0x31 & 49  & 0x4E & 78  & 0xB1 & 177 \\ \hline
0x3D & 61  & 0x37 & 55  & 0x07 & 7   \\ \hline
0x43 & 67  & 0x49 & 73  & 0x94 & 148 \\ \hline
0x49 & 73  & 0xFF & 255 & 0x05 & 5   \\ \hline
0x53 & 83  & 0xFF & 255 & 0x02 & 2   \\ \hline
0x81 & 129 & 0xFF & 255 & 0x10 & 16  \\ \hline
0x83 & 131 & 0xFF & 255 & 0x07 & 7   \\ \hline
0x85 & 133 & 0xFF & 255 & 0x06 & 6   \\ \hline
\end{longtable}
}

\label{tab:verificacao}

Além da tabela numérica, a Figura~\ref{fig:ondas} mostra a forma de onda obtida 
na simulação do testbench no QuestaSim. 

Observa-se o sinal de entrada \texttt{scancode\_in}, e as saídas 
\texttt{ascii\_cuv} e \texttt{ascii\_golden}. 
Nos instantes correspondentes aos scancodes divergentes, os sinais de saída não coincidem, 
confirmando as diferenças listadas na Tabela~\ref{tab:verificacao}.

\begin{figure}[h!]
    \centering
\includegraphics[width=\textwidth]{images/waveform.png}
    \caption{Divergências entre o CUV e o Golden Model.}
    \label{fig:ondas}
\end{figure}

\section{Conclusão}
A verificação funcional mostrou que:
\begin{itemize}
    \item O Golden Model implementa corretamente a especificação.
    \item O CUV apresentou divergências em determinados scancodes.

\end{itemize}


\end{document}
